%_______________________________________________________________________________________________________________________________%
%*******************************************************************************************************************************%
%===============================================================================================================================%
%-------------------------------------------------- Dissertation-Prologue.tex --------------------------------------------------%
%===============================================================================================================================%
%*******************************************************************************************************************************%
% File: Dissertation-Prologue.tex
% Description: this file is where the abstract, acknowledgements, dedication, and attributions should be defined AND/OR excluded.
% Usage: + Fill in the \X{...} macros that you want to be INCLUDED in your document; this automatically sets the corresponding
%        \if@makeXPage conditional to TRUE and inserts the associated \thesisXPage into the document in the proper order.     
%        + To EXCLUDE any of these frontmatter pages from the document, uncomment the corresponding \excludeX macros. Commenting
%        out the \X{...} macro will also prevent the corresponding Page from being included in the document.
%
%
%_______________________________________________________________________________________________________________________________%
% Last Updated in December, 2023
%_______________________________________________________________________________________________________________________________%
%===============================================================================================================================%
%------------------------------------------------------ Prologue Controls ------------------------------------------------------%
%===============================================================================================================================%
%\makeCopyrightPage
%\makeListOfFigures
%\makeListOfTables
%-------------------------------------------------------------------------------------------------------------------------------%
%\excludeAbstractPage
%\excludeAcknowlegmentsPage
%\excludeDedicationPage
%\excludeAttributionsPage
%\emptyLoT                                                              % Tell latex that there is no list of tables
%-------------------------------------------------------------------------------------------------------------------------------%
%\setUppercaseHeadings
%\setCapitalizedHeadings
%-------------------------------------------------------------------------------------------------------------------------------%
%\setUppercaseToCtitles
%\setCapitalizedToCtitles
%_______________________________________________________________________________________________________________________________%
%*******************************************************************************************************************************%
%===============================================================================================================================%
%--------------------------------------------------------- FRONTMATTER ---------------------------------------------------------%
%-------------------------------------------------------------------------------------------------------------------------------%
%----------------------------------------- Frontmatter (Prologue) page(s) text content -----------------------------------------%
%===============================================================================================================================%
%*******************************************************************************************************************************%
%_______________________________________________________________________________________________________________________________%
%===============================================================================================================================%
%---------------------------------------------------------- Abstract -----------------------------------------------------------%
%===============================================================================================================================%
\abstract{The work presented in this dissertation...
}
%===============================================================================================================================%
%------------------------------------------------------ Acknowledgements -------------------------------------------------------%
%===============================================================================================================================%
\acknowledgements{Special thanks go to...
}
%===============================================================================================================================%
%--------------------------------------------------------- Dedication ----------------------------------------------------------%
%===============================================================================================================================%
%-------------------------------------------------------------------------------------------------------------------------------%
%\setDedicationFontstyle{it}
%\setPageFontstyle{Dedication}{bf}
%-------------------------------------------------------------------------------------------------------------------------------%
\dedication{This is dedicated to...
}
%===============================================================================================================================%
%-------------------------------------------------------- Attributions ---------------------------------------------------------%
%===============================================================================================================================%
%-------------------------------------------------------------------------------------------------------------------------------%
% Example usage of the \attributions macro and the associated 'attributionList', 'attribution', and 'authorList' environments
% designed to generate an appropriately formatted attributions list.
%-------------------------------------------------------------------------------------------------------------------------------%
\attributions{%
\begin{attributionList}
%-------------------------------------------------------------------------------------------------------------------------------%
    \item\begin{attribution}
        %Schultze, B. E., M. Witt, Y. Censor, K. E. Schubert, and R. W. Schulte (2015). Performance of hull-detection algorithms for proton computed tomography reconstruction. In S. Reich and A. Zaslavski (Eds.), \textit{Infinite Products of Operators and Their Applications}, Volume 636 of \textit{Contemporary Mathematics}, pp. 211–224. American Mathematical Society.
         B. E. Schultze, M. Witt, Y. Censor, K. E. Schubert, and R. W. Schulte, “Performance of hull-detection algorithms for proton computed tomography reconstruction,” in \textit{Infinite Products of Operators and Their Applications}, ser. Contemporary Mathematics, S. Reich and A. Zaslavski, Eds., vol. 636. American Mathematical Society, 2015, pp. 211–224.
    \end{attribution}\label{attribution:hull-detection}%
    \begin{authorList}
        \item I (B. E. Schultze) was the sole investigator and primary author of the publication.
        \item M. Witt provided the simulated data sets used for the initial hull-detection investigations.
        \item Y. Censor, K. E. Schubert, and R. W. Schulte acted in a supervisory role on editing, content accuracy, and approval of the final form submitted for publication.
    \end{authorList}
%-------------------------------------------------------------------------------------------------------------------------------%
    \item\begin{attribution}%
        %Schultze, B. E., Y. Censor, P. Karbasi, K. E. Schubert, and R. W. Schulte (2020). An Improved Method of Total Variation Superiorization Applied to Reconstruction in Proton Computed Tomography. \textit{IEEE Transactions on Medical Imaging 39}(2), 294–307.
        B. E. Schultze, Y. Censor, P. Karbasi, K. E. Schubert, and R. W. Schulte, “An Improved Method of Total Variation Superiorization Applied to Reconstruction in Proton Computed Tomography,” \textit{IEEE Transactions on Medical Imaging}, vol. 39, no. 2, pp. 294–307, 2020
    \end{attribution}\label{attribution:ntvs}
    \begin{authorList}
        \item I (B. E. Schultze) was the sole investigator and primary author of the publication.
        \item Y. Censor is among the original developers of the superiorization methodology, which serves as the theoretical framework of total variation superiorization, and requested the investigations be performed for pCT. He also provided approval of the investigation results and final form submitted for publication.
        \item P. Karbasi was a colleague working independently on a related topic who participated in discussions to ensure our investigations did not overlap. She also assisted in the editing of the final form submitted for publication.
        \item K. E. Schubert and R. W. Schulte acted in a supervisory role on editing, content accuracy, and approval of the final form submitted for publication.%
    \end{authorList}%
\end{attributionList}
}
%_______________________________________________________________________________________________________________________________%
%*******************************************************************************************************************************%
%===============================================================================================================================%
%----------------------------------------------- END: Dissertation-Prologue.tex ------------------------------------------------%
%===============================================================================================================================%
%*******************************************************************************************************************************%
\endinput
